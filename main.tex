\documentclass[a4paper]{article}
%\usepackage{simplemargins}

%\usepackage[square]{natbib}
\usepackage{amsmath}
\usepackage{amsfonts}
\usepackage{amssymb}
\usepackage{graphicx}
\usepackage{mathtools}
\begin{document}
\pagenumbering{gobble}

\Large
 \begin{center}
 Performance analysis of Optically Pumped ${}^{4}\mathbf{He}$ Magnetometers vs. Conventional SQUID\\ 

\hspace{10pt}

% Author names and affiliations
\large
Arthur Author$^1$, Cecilia CoAuthor$^2$ \\

\hspace{10pt}

\small  
$^1$ First affiliation\\
arthur.author@correspondence.email.com\\
$^2$ Second affiliation

\end{center}

\hspace{10pt}

\normalsize
\textbf{Background} / \\

Optically-pumped magnetometers (OPMs) have recently reached sensitivity levels required for magnetoence-phalography (MEG). OPMs do not need cryogenics and can thus be placed within millimetres from the scalpinto an array that adapts to the individual head size and shape, thereby reducing the distance from cortical sources to the sensors.\\
Optically Pumped Magnetometers (OPMs) have emerged as a viable and wearable alternative to cryogenic, superconducting MEG systems. This new generation ofsensors has the advantage of not requiring cryogenic cooling and as a result can beflexibly placed on any part of the body. \\

Unfortunately, the SQUID-based systems have some limitations related to the need to cool them down with liquid helium. The room-temperature alternatives for SQUIDs are optically pumped magnetometers (OPM) operating in spin exchange relaxation-free (SERF) regime, which require a very low ambient magnetic field.\\

OPMs operate on very different physical principles to SQUIDs. Prin-cipally, they do not require cryogenic cooling and can be placed withinmillimetres of the subject's scalp. This simple advantage of bringing thesensors closer to the subject's brain offers a three-tofive-fold improve-ment in sensitivity (as well as consistency across different headshapesand sizes) to cortical sources compared to traditional SQUID based MEG \\

Optically Pumped Magnetometers (OPMs) are capable of measuring very weak magneticfields (femtotesla sensitivity) without the need for cryogenic cooling (Shah and Wakai, 2013). This ability means that OPMscan be used to develop a new generation of more flexible and highly sensitive Magnetoencephalography (MEG) systems \\

We demonstrate the first use of Optically Pumped Magnetoencephalography (OP‐MEG) in an epilepsy patient with unrestricted head movement. Current clinical MEG uses a traditional SQUID system, where sensors are cryogenically cooled and housed in a helmet in which the patient’s head is fixed. Here, we use a different type of sensor (OPM), which operates at room temperature and can be placed directly on the patient’s scalp, permitting free head movement. 

\textbf{Methods} /  \\

 Here, we quantified the improvement in recording MEG with hypothetical on-scalp OPMarrays compared to a 306-channel state-of-the-art SQUID array (102 magnetometers and 204 planargradiometers).We simulated OPM arrays that measured either normal (nOPM; 102 sensors), tangential (tOPM; 204sensors), or all components (aOPM; 306 sensors) of the magneticfield. We built forward models based on magnetic resonance images of 10 adult heads; we employed a three-compartment boundary element model anddistributed current dipoles evenly across the cortical mantle.
 
 ompared to the SQUID magnetometers, nOPM and tOPM yielded 7.5 and 5.3 times higher signal power,while the correlations between the field patterns of source dipoles were reduced by factors of 2.8 and 3.6,respectively. Values of the field-pattern correlations were similar across nOPM, tOPM and SQUID gradiometers.

The OPM and MEG forward models were based on anatomical data from a 42-year old male volunteer subject and 9 yrs ....\\

The MEG sensor array consisted of N = 102 MEG magnetometers, measuring the magnetic field component approximately radial to the surface of the head; The OPM sensor array that measured either normal (nOPM; 102 sensors), tangential (tOPM; 204sensors), or all components (aOPM; 306 sensors) of the magneticfield.\\

The forward matrix was computed using the Brainstorm software

We quantified computationally the dependency of MEG and OPM on the source orientation using a forward model with realistic tissue boundaries.\\

In the present simulation study we mapped the dependency of OPM and MEG signal magnitude on the orientation of a current dipole in a dense grid of source locations in the human cerebral cortex. We examined the ratio of the signals due to a dipole having an orientation with the lowest vs. the highest sensitivity for an array of OPM or MEG sensors. A forward solution for each source location was constructed using a boundary-element method. The source orientations with the lowest and the highest sensitivity were determined with the help of the singular value decomposition (SVD). 

the SVD automatically provides the source
orientation of lowest and highest signal power in the sensor
array.


 currents for tOPM

\textbf{Results} /  \\

The median value for the ratio of the signal magnitude for the source orientation of the lowest and the highest sensitivity was 0.06 for MEG and *0.63° for OPM. 

 Volume currents reduced the signals of primary currents on average by 10\% , 72\% and 15\% in nOPM, tOPM andSQUID magnetometers, respectively. The information capacities of the OPM arrays were clearly higher than that of the SQUID array. The dipole-localization accuracies of the arrays were similar while the minimum-norm-based point-spread functions were on average 2.4 and 2.5 times more spread for the SQUID array compared to nOPM and tOPM arrays, respectively.\\
 
 
 On average, the topographypowers of the nOPM and tOPM are 7.5 and 5.3 times higher than the topography power of mSQUID. The topography power of the nOPM arrayis on average 1.5 times higher than that of tOPM.\\

The overall amplitudes of the volume current topographies are of the same magnitude as those of the primary. The field due to volume currents substantially suppresses the overall amplitude of the primary-current topography in tangential measurements. For nOPM and mSQUID, the overall magnitude of the primary current topography is much higher than that of volume currents. Additionally, volume currents do not result in a major decrease in the overall amplitude of primary current topography in nOPM and mSQUID. Furthermore, the closer the normal-component measuring sensors are to the sources, the larger the overall magnitude of the primary-current topography is relative to the volume-current topography\\

The total information capacities of the arrays are presented in. Both of the OPM arrays (nOPMand tOPM) provide more information than aSQUID. tOPM yields more information than nOPM and the combined array aOPM even more. The total information capacity per sensor is highest in nOPM

\textbf{ Discussion} / \\



\end{document}